\documentclass[magyar]{beamer}
\usepackage[T1]{fontenc}
\usepackage[utf8]{inputenc}
\usepackage{verbatim}
\usepackage{beamerthemeboxes}
\PassOptionsToPackage{normalem}{ulem}
\usepackage{ulem}
\usetheme{Madrid}
\usepackage{tikz} 
\usetikzlibrary{shapes}

\makeatother

\usepackage{babel}

\defbeamertemplate*{footline}{mytheme}
{
  \leavevmode%
  \hbox{%
  \begin{beamercolorbox}[wd=.5\paperwidth,ht=2.25ex,dp=1ex,center]{author in head/foot}%
	    \usebeamerfont{author in head/foot}{Nagy Zoltán Arnold}~~
  \end{beamercolorbox}%
  \begin{beamercolorbox}[wd=.5\paperwidth,ht=2.25ex,dp=1ex,right]{date in head/foot}%
    \usebeamerfont{date in head/foot}{EuroBSDCon 2011}\hspace*{2em}
    \insertframenumber{} / \inserttotalframenumber\hspace*{2ex}
  \end{beamercolorbox}}%
  \vskip0pt%
}

\usebeamertemplate{mytheme}
\date{}
\begin{document}

\title{NPF: a new packet filter}
\author{Zoltán Arnold Nagy\\ Mindaugas Rasiukevicius \\ The NetBSD Foundation \\ \{zoltan,rmind\}@netbsd.org}
\maketitle

\begin{frame}
\begin{itemize}
	\item What is a packet filter?
\pause
	\begin{itemize}
		\item A boolean valued function ($f:\mathbb{P} \rightarrow \mathbb{B}$)
	\end{itemize}
\pause
	\item How can it be implemented?
\pause
	\begin{itemize}
		\item boolean expression tree
		\item directed acyclic control flow graph
	\end{itemize}
\end{itemize}
\end{frame}

\begin{frame}
\frametitle{A see of firewalls}
\begin{itemize}
	\item IPFilter (ipf)
	\item FreeBSD's ipfw
	\item OpenBSD's pf
	\item NetBSD's npf
\end{itemize}
\end{frame}

\begin{frame}
\frametitle{How NPF started out}
\begin{itemize}
	\item Sponsored by The NetBSD Foundation
	\item Written by Mindaugas Rasiukevicius (rmind@) from scratch, altought the design was inspired by the Berkeley Packet Filter
	\item First imported to -current in August 2010
	\item Will be widely available with the 6.0 release
	\item First step in improving NetBSD's networking capabilities
	\item Second step: removing the big kernel lock (proposal deadline: Oct 31st)
\end{itemize}
\end{frame}

\begin{frame}
\frametitle{Google Summer of Code}
\begin{itemize}
	\item Goal: The program's goal is to promote open-isource software development among students and get them actually involved
\pause
	\item Mentoring organizations apply with a set of proposed projects
\pause
	\item Students apply for organizations with a proposal based either on a proposed project or a new one
\pause
	\item Organizations get a number of slots for students
\pause
	\item Mentoring organizations get \textdollar500 per student, students get \textdollar5000
\pause
	\item It has been running since 2005
		\begin{itemize}
			\item 2011: 175 mentoring organizations, 1115 students
		\end{itemize}
	\item NetBSD has participated every year so far with a high success rate
		\begin{itemize}
			\item 2011: 9 projects, 8 projects ended with success
		\end{itemize}
	\item My GSoC proposal for 2011 was to add IPv6 support to NPF
		\begin{itemize}
			\item currently not in -current (soon!)
			\item available from \url{https://github.com/zoltan/ipv6-npf}
		\end{itemize}
\end{itemize}
\end{frame}

\begin{frame}
\frametitle{Motivations and goals}
\begin{itemize}
	\item There a few existing firewalls
	\item It's easier to design a new firewall from ground up than to clean up existing codebases
	\item Design goals for NPF:
	\begin{itemize}
		\item MP-safety and locklessness for scalable MP performance
		\item Fast tree- and hash-based lookup support for tables
		\item Stateful packet filtering
		\item N-Code processor, a general bytecode engine
		\item Keep configuration syntax changes to a minimum
		\item Modularity, extensibility: an extension API for developers, hooking support
		\item Last but not least: simplicity
	\end{itemize}
	\item Of course it's portable, uses pfil(9) hooks; DragonFlyBSD is considering adoption 
\end{itemize}
\end{frame}

\begin{frame}
\frametitle{What can it do today?}
\begin{itemize}
	\item Syntax is nearly identical to other firewalls
	\item Group support
	\item Rule procedures (connection-based packet transformations)
	\begin{itemize}
		\item IP ID randomization
		\item enforcement of TCP minimum TTL
		\item enforcement of TCP Maximum Segment Size (MSS)
		\item logging
	\end{itemize}
	\item Tables support
\end{itemize}
\end{frame}

\begin{frame}[fragile,basicstyle=\ttfamily]
\begin{verbatim}
ext_if = "wm0"
ext_if = "wm1"
table "1" type "tree" dynamic
procedure "rid" { normalize (random-id) }
procedure "log" { log npflog0 }
group (name "external", interface $ext_if) {
    block in quick from <1>
    pass out quick from $ext_if keep state apply "rid"
    pass in quick proto tcp to $ext_if port ssh apply "log"
    ...
}
group (name "internal", interface $int_if) {
    block in all
    pass in quick from <1>
    pass out quick all
}
group (default) { block all }
\end{verbatim}
\end{frame}

\begin{frame}
\frametitle{Inside}
N-code engine:
\begin{itemize}
	\item General purpose bytecode engine, 32-bit words, 4 registers available
	\item The firewall configuration is compiled to our bytecode format, then loaded
	\item CISC and RISC-like instructions
	\item The packets are processed as a byte-stream
\end{itemize}
Efficient internal structures
\begin{itemize}
	\item npf\_addr (in6\_addr, 128-bit) for addresses (the first 32 bit is used for IPv4 addresses)
	\item uint8\_t for masks
	\begin{itemize}
		\item instead of generating the appropriate npf\_addr value from  255.255.255.0 or /24, we just store
the mask (<=128)
		\item tradeoff: CPU for memory
	\end{itemize}
\end{itemize}
\end{frame}

\begin{frame}
\frametitle{Current TODO list}
\begin{itemize}
	\item Replace red-black trees with Patricia (radix) trees for tables
\pause
	\item Pregenerate an array for masks
\pause
	\item Support per-rule statistics
\pause
	\item Multiple address per interface support on interface-based rules
	\item Dynamic interface tracking
		\begin{itemize}
			\item Let's say you have a rule based on an interface instead of an address...
			\item ...you compile and load the rules...
			\item ...then change the interface's address?
		\end{itemize}
	\item Control fragmentation on a per-interface basis
\pause
	\item Stateful NAT64 [RFC 6146]
	\item IPv6-to-IPv6 Network Prefix Translation (NPTv6) [RFC 6296]
\pause
	\item Write more documentation :)
\end{itemize}
\end{frame}

\begin{frame}
\begin{center}
Questions and answers?
\end{center}
\end{frame}

\begin{frame}
\begin{center}
Thank you!
\end{center}
\end{frame}

\end{document}
