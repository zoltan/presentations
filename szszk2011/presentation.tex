\documentclass[magyar]{beamer}
\usepackage[T1]{fontenc}
\usepackage[utf8]{inputenc}
\usepackage{verbatim}
\usepackage{beamerthemeboxes}
\PassOptionsToPackage{normalem}{ulem}
\usepackage{ulem}
\usetheme{Madrid}
\usepackage{tikz} 
\usetikzlibrary{shapes}

\makeatother

\usepackage{babel}

\defbeamertemplate*{footline}{mytheme}
{
  \leavevmode%
  \hbox{%
  \begin{beamercolorbox}[wd=.5\paperwidth,ht=2.25ex,dp=1ex,center]{author in head/foot}%
	    \usebeamerfont{author in head/foot}{Nagy Zoltán Arnold}~~
  \end{beamercolorbox}%
  \begin{beamercolorbox}[wd=.5\paperwidth,ht=2.25ex,dp=1ex,right]{date in head/foot}%
    \usebeamerfont{date in head/foot}{Szabad Szoftver Konferencia 2011}\hspace*{2em}
    \insertframenumber{} / \inserttotalframenumber\hspace*{2ex}
  \end{beamercolorbox}}%
  \vskip0pt%
}
\usebeamertemplate{mytheme}
\begin{document}

\title{NPF: A NetBSD új tűzfala}
\author{Nagy Zoltán Arnold \\ The NetBSD Foundation \\ zoltan@netbsd.org}
\maketitle

\begin{frame}
\frametitle{Google Summer of Code}
\begin{itemize}
	\item A cél a szabad szoftverek támogatása
	\item Idén 7 éves a program
	\item Nyitott minden egyetemi hallgató számára
	\item A NetBSD minden évben ott volt, idén 9 projekt
\end{itemize}
\end{frame}

\begin{frame}
\frametitle{Tűzfalkörkép}
\begin{itemize}
	\item IPFilter
	\item ipfw
	\item pf
	\item npf
\end{itemize}
\end{frame}

\begin{frame}
\frametitle{Az NPF-ről}
\begin{itemize}
	\item A NetBSD Foundation által szponzorált projekt
	\item Az eredeti kódot Mindaugas Rasiukevicius (rmind@) írta
	\item Idén kiegészítettem illetve bekapcsolódtam a fejlesztésbe
	\item A NetBSD 6.0 -ban lesz széles körben elérhető
	\item Az első állomás a NetBSD hálózati kódjának javításában
\end{itemize}
\end{frame}

\begin{frame}
\frametitle{Motiváció és célok}
\begin{itemize}
	\item A mostani tűzfalkódok többségét tervezés nélkül bővítették
	\item A mi tervezési céljaink:
	\begin{itemize}
		\item minél inkább lockless kód
		\item ebből adódó SMP skálázódás
		\item állapottartó szűrés
		\item modularitás és bővíthetőség
		\item általános bytecode motor
	\end{itemize}
\end{itemize}
\end{frame}

\begin{frame}
\frametitle{Mit tud jelenleg?}
\begin{itemize}
	\item Tetszőleges definíciók (ext\_if = "wm0")
	\item Csoportok (group (name "external", interface \textdollar ext\_if)
	\item Rule procedures (csomagtranszformációk egy kapcsolatra)
	\begin{itemize}
		\item IP ID randomizálás
		\item TCP minimum TTL betartatás
		\item TCP Maximum Segment Size (MSS) betartatás
		\item Loggolás
	\end{itemize}
	\item Tábla támogatás
\end{itemize}
\end{frame}

\begin{frame}
\frametitle{Belülről}
N-code engine:
\begin{itemize}
	\item Általános célú bytecode engine, 32-bit minden szó, 4 regiszter
	\item A tűzfalszabályok erre a bytecodera fordulnak le
	\item CISC és RISC szerű utasítások
	\item Maga a csomag nem kerül értelmezésre, csak mint bytefolyam
\end{itemize}
Okos ábrázolási módok
\begin{itemize}
	\item in6\_addr (128-bit) a címekhez (első 32 bit ha IPv4)
	\item uint8\_t a maszkokhoz
\end{itemize}
\end{frame}

\begin{frame}
\frametitle{Fejlesztési célok}
\begin{itemize}
	\item Piros-fekete fa -> Radix fa a táblákhoz
	\item Maszkokat előre legenerálni
	\item Szabályonkénti statisztikák támogatása
	\item Dinamikus interfészkövetés
	\item Plugin támogatás bevezetése
\end{itemize}
\end{frame}

\begin{frame}
\begin{center}
Kérdések?
\end{center}
\end{frame}

\end{document}
